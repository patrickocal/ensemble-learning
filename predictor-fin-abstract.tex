% !TEX TS-program = pdflatexmk
 % 20230101 

%In the setting of \citet{GS_Inductive_inference}, we weaken the diversity axiom
%to allow us to account We provide an axiomatic derivation of prices in the
%market for Zeros (zero-coupon bonds derived from US Treasuries).
%
%A learner (prediction model) is stable if it generalizes from the current
%sample of past cases to samples that include novel case types without revising
%the current kernel. 
%A pricing kernel is stable if neither (i) the arrival of a
%novel case type, nor (ii) the threat of arbitrage cause a re-evaluation of the
%empirical weights it assigns to past cases.  A stable learner that forecasts
%according to the empirical mean also satisfies the statistical learning notion
%of stability: continuity of the learning map from samples to learners.


%A learner is \emph{stable} if it meaningfully generalizes
%from the current sample of past cases to richer samples (novel case types)
%without revision of the current kernel.
%This implies the statistical learning
%notion of stability: a stable learner's prediction error tends to zero as the
%sample size grows. In market settings, stable pricing kernels support the
%efficient markets hypothesis: frequent cases exhibit diminishing proportional
%returns. Our canonical setting is the market for Zeros (zero-coupon bonds
%derived from US Treasuries) where the (pricing) kernel is a function of the
%annual yield.
%In addition to stability, we require \parthreediv\ (for every
%three maturity dates, past cases generate at least $3$ total rankings): this
%holds robustly on (subsamples of) 1961-2023 daily data whereas other similar
%conditions \citep[for example]{GS_Inductive_inference,GLS-empirical-similarity}
%do not.
%  Stable learners are characterised by the following two equivalent
%  properties: given any sample of past cases, (i) the yield accumulation
%  process is arbitrage-free (ii) the pricing kernel is the canonical empirical
%  geometric mean (with subjective yields). There are therefore three routes to
%  stability: sufficiently rich data; out-of-sample validatation; no-Dutch-books
%  validation.


%\begin{enumerate}[label=\roman*]
%
%\item acquire a sufficiently rich set of past cases. For example, the
%\fourdiv\ axiom of \citet{GS_Inductive_inference}  (for every four maturity
%dates, past cases generate $4!=24$ rankings of the corresponding annual
%yields).  
%
%\item conditional on any sample of past cases, the pricing kernel is arbitrage
%free.  (This is reminiscent of De Finetti's Dutch book argument for
%conditional probabilities.)
%
%\item assume \parthreediv\ (every three maturity dates generate $3$ rankings)
%and assume some form \end{enumerate} Even with 61 years of daily data,
%\fourdiv\ fails to hold.  The second and third routes to stability require
%some form of external validation or imagination to compensate for the lack of
%rich data.  Our \parthreediv\ (every three maturity dates generate $3$
%rankings) holds after five years.  The third and final 
%
%In the absence of sufficiently rich data, 
%
%
%  external validation.  Ensuring the price kernel is arbitrage free
%
%We provide axioms for stable learning that extend 
%
%Moreover, we extend the axiomatic framework of \citet{GS_Inductive_inference}
%to account for stable learning when the data These two routes to stable
%learning are 
%
%
A learner (prediction model) is \emph{stable} if it meaningfully generalizes
from current samples of past cases to richer one (containing novel case types)
without revision of the current kernel.
%(This notion is sufficient for the statistical learning one.)
On the basis of sentiments (a map from samples to rankings), we establish three
kinds of stable learner.  The first relies on the richness of past cases (\eg\
\fourdiv, for every four eventualities, past cases generate at least $4!=24$
total rankings).  The second relies on prudence (using the imagination or
out-of-sample validation to explore the impact of novel case types on their
model) and \parthreediv\ (for every three eventualities, past cases generate at
least $3$ total rankings). The third arises when, as a market maker, the
learner bypasses their sentiments to ensure their pricing kernel is
arbitrage-free.  The stable learner's sentiments are characterised by two
equivalent properties (i) their kernel is the canonical empirical mean with
subjective weights (yields in a market setting) and (ii) a groupoid condition
(no-arbitrage in market settings).  Stable pricing kernels support the
efficient markets hypothesis since the marginal impact of an additional case is
decreasing in the relative frequency of its type.  Active strategies rely on
the ability to predict instability (structural breaks) or rare cases.  For
zero-coupon treasury bond daily data over 1961-2023, \parthreediv\ holds
whereas \fourdiv\ does not.


%prudent market makers: no-arbitrage equilibrium prices behave as if guided by a
%prudent learner.  
%
%
%
%% No-arbitrage pricing ensures implied market yields extend consistently to
%% novel case types.
%
% To such settings, we formally extend \cite{GS_Inductive_inference}.

 %Three learners are asked
 %to rank investments according to their plausibility, given any finite
 %resampling of (observable) past cases.
 %The first and second only consider past cases, but the second also ensures
 %the
 %The second uses their imagination to explore (unobservable) novel
 %cases whenever resampling past cases fails to generate a full
 %diversity of rankings. The third ensures their A learner is stable if their model generalizes to new
 %data.
 % As a market maker for zero-coupon bonds, the prudent learner 
 %is distinguished by two equivalent properties: conditional on any finite
 %resampling of past cases, 
 %(i) she sets prices according to the canonical empirical geometric mean (with
 %subjective yields) and 
 %(ii) the associated accumulation process is arbitrage-free.
 %A novel form of market efficiency emerges even in the absence of prudent
 %market makers:
 %no-arbitrage equilibrium prices behave as if guided by a prudent learner.
%% No-arbitrage pricing ensures implied market yields extend consistently to novel case types.

 %To such settings, we formally extend \cite{GS_Inductive_inference}.
