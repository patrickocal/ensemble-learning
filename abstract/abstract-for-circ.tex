\documentclass{article}
\begin{document}
\title{Stable learning and no-arbitrage pricing on the basis of sentiments}
  A stable learner is a prediction model that generalises to new samples without re-evaluation of the current kernel.  In market settings, a learner
  is stable if, and only if, its pricing kernel is arbitrage-free.  Stable pricing 
  kernels preserve information and support the efficient markets hypothesis
  since the marginal impact of the next observation is decreasing in the relative
  frequency of its type. Our contribution is to derive the current kernel on the
  basis of sample-based sentiments (rankings) for a broader class of stable learner.  
  This extension of Gilboa and
  Schmeidler (2003) is justified by US Treasury bond market data for the period
  1961-2023. Our ``partial-3-diversity of data'' condition on daily rankings of 
  annual yields holds across 91\% of the sample for all 30 maturity dates.  When
  stronger diversity conditions fail, stability requires out-of-sample validation:
  sentient agents might simulate or imagine the impact of novel data; artificial
  ones might engage in leave-one-type-out cross validation; or, one might bypass sentiments and directly rule out Dutch books. Our
  framework embeds current sentiments in a system of potential generalisations
  thus enabling within-sample testing of stability for any form of external validation.

\end{document}